\documentclass{beamer}
\usepackage[utf8]{inputenc}
\usepackage{graphicx}

\newtheorem{definicion}{Definición}
\newtheorem{ejemplo}{Ejemplo}

%%%%%%%%%%%%%%%%%%%%%%%%%%%%%%%%%%%%%%%%%%%%%%%%%%%%%%%%%%%%%%%%%%%%%%%%%%%%%%%

  
\title[Presentación con Beamer]{Número PI }
\author[Elízabeth Díez Rodríguez]{Elízabeth Díez Rodríguez}
\date[23-04-2014]{23 de abril de 2014}
%%%%%%%%%%%%%%%%%%%%%%%%%%%%%%%%%%%%%%%%%%%%%%%%%%%%%%%%%%%%%%%%%%%%%%%%%%%%%%%

%\usetheme{Madrid}
%\usetheme{Antibes}
%\usetheme{tree}
%\usetheme{classic}


\begin{document}
%++++++++++++++++++++++++++++++++++++++++++++++++++++++++++++++++++++++++++++++
\begin{frame}
  \titlepage
  \begin{small}
    \begin{center}
     Facultad de Matemáticas \\
     Universidad de La Laguna
    \end{center}
  \end{small}

\end{frame}
%++++++++++++++++++++++++++++++++++++++++++++++++++++++++++++++++++++++++++++++
\begin{frame}
  \frametitle{Índice}
  \tableofcontents[pausesections]
\end{frame}
%++++++++++++++++++++++++++++++++++++++++++++++++++++++++++++++++++++++++++++++

%++++++++++++++++++++++++++++++++++++++++++++++++++++++++++++++++++++++++++++++


\section{Valor de $\pi$}


%++++++++++++++++++++++++++++++++++++++++++++++++++++++++++++++++++++++++++++++
\begin{frame}

\frametitle{Valor de $\pi$}

\begin{definicion}
El valor de $\pi$ se ha obtenido con diversas aproximaciones a lo largo de la historia,
siendo una de las constantes matemáticas que más aparece en las ecuaciones de la física, junto con el número e.
Cabe destacar que el cociente entre la longitud de cualquier circunferencia y 
la de su diámetro no es constante en geometrías no euclídeas.


\end{definicion}

\end{frame}
%++++++++++++++++++++++++++++++++++++++++++++++++++++++++++++++++++++++++++++++

\section{nombre de $\pi$}

%++++++++++++++++++++++++++++++++++++++++++++++++++++++++++++++++++++++++++++++
\begin{frame}

\frametitle{nombre de $\pi$}

\begin{definicion}
La notación con la letra griega $\pi$ proviene de la inicial de las palabras de origen griego $\pi$epilyepela 'periferia'
y $\pi$epilyepela 'perímetro' de un círculo,1 notación que fue utilizada primero por William Oughtred (1574-1660) y 
cuyo uso fue propuesto por el matemático galés William Jones2 (1675-1749); aunque fue el matemático Leonhard Euler,
con su obra Introducción al cálculo infinitesimal, de 1748, quien la popularizó. Fue conocida anteriormente como constante
de Ludolph (en honor al matemático Ludolph van Ceulen) o como constante de Arquímedes (que no se debe confundir con el número
de Arquímedes).

\end{definicion}

\end{frame}

%++++++++++++++++++++++++++++++++++++++++++++++++++++++++++++++++++++++++++++++
\section{FORMULAS}

\begin{frame}
\frametitle{FORMULAS}



\begin{block}{FORMULAS QUE CONTIENEN EL NUMERO PI}
  \begin{itemize}
    \item <1-> Primera.Longitud de la circunferencia de radio:
    r: C = 2* $\pi$ * r
\pause
    \item <2-> Segunda.Área del círculo de radio r: A = $\pi$ $r^2$
    \pause
    \item <3-> Tercera.
    x=\frac{a_2 x^2 + a_1 x + a_0}{1+2z^3}, \quad
x+y^{2n+2}=\sqrt{b^2-4ac} \pause
    \item <4-> Cuarta.
    S_n=a_1+\cdots + a_n = \sum_{i=1}^n a_i \pause
    \item <4-> Quinta.
    \int_{x=0}^{\infty} x\,\text{e}^{-x^2}
\text{d}x=\frac{1}{2},\quad\text{e}^{i\pi}+1=0 \pause
  \end{itemize}
\end{block}

\end{frame}


%++++++++++++++++++++++++++++++++++++++++++++++++++++++++++++++++++++++++++++++

\section{Bibliografía}
%++++++++++++++++++++++++++++++++++++++++++++++++++++++++++++++++++++++++++++++
\begin{frame}
  \frametitle{Bibliografía}

  \begin{thebibliography}{10}


    \beamertemplatebookbibitems
    \bibitem[numero pi]{}
    Numero pi
   
    {\small http://es.wikipedia.org}

    \beamertemplatebookbibitems
    \bibitem[numerp pi]{latex}
    pi. {\small http://webs.adam.es/rllorens/pi.htm }

  \end{thebibliography}
\end{frame}

%++++++++++++++++++++++++++++++++++++++++++++++++++++++++++++++++++++++++++++++
\end{document}